\documentclass[10pt,journal,letterpaper,compsoc]{IEEEtran}

\usepackage{cite}
\usepackage[dvips]{graphicx}
\usepackage{algorithm}
\usepackage{algpseudocode}
\usepackage{array}
\usepackage{float}
\usepackage[tight,footnotesize]{subfigure}
\usepackage{url}
\usepackage{multirow}
\newcommand\du{\mathrm{d}}

\usepackage[cmex10]{amsmath}
\interdisplaylinepenalty=2500

% \hyphenation{op-tical net-works semi-conduc-tor}


\begin{document}

\title{Ensuring the performance of Apache Web Server affected by aging}

\author{Jing~Zhao, Kishor~S.~Trivedi, Yanbin~Wang%
\IEEEcompsocitemizethanks{\IEEEcompsocthanksitem Jing Zhao is with the
Department of Computer Science and Technology, Harbin Engineering
University, Harbin, China, 150001\protect\\
E-mail: jingzhao.duke@gmail.com
\IEEEcompsocthanksitem Kishor S. Trivedi is with the Department of
Electrical and Computer Engineering, Duke University, Durham, USA, 27708
\IEEEcompsocthanksitem Yanbin Wang is with the Department of
Industrial Engineering, Harbin Institude of Technology,
Harbin, China, 150001}%
\thanks{}}

% The paper headers
\markboth{IEEE Transactions on Dependable and Secure Computing,~Vol.~?, No.~?, May~2012}%
{Jing Zhao \MakeLowercase{\textit{et al.}}: Ensuring the performance of Apache Web Server affected by aging}
% The only time the second header will appear is for the odd numbered pages
% after the title page when using the twoside option.
% 
% *** Note that you probably will NOT want to include the author's ***
% *** name in the headers of peer review papers.                   ***
% You can use \ifCLASSOPTIONpeerreview for conditional compilation here if
% you desire.



% The publisher's ID mark at the bottom of the page is less important with
% Computer Society journal papers as those publications place the marks
% outside of the main text columns and, therefore, unlike regular IEEE
% journals, the available text space is not reduced by their presence.
% If you want to put a publisher's ID mark on the page you can do it like
% this:
%\IEEEpubid{0000--0000/00\$00.00~\copyright~2007 IEEE}
% or like this to get the Computer Society new two part style.
%\IEEEpubid{\makebox[\columnwidth]{\hfill 0000--0000/00/\$00.00~\copyright~2007 IEEE}%
%\hspace{\columnsep}\makebox[\columnwidth]{Published by the IEEE Computer Society\hfill}}
% Remember, if you use this you must call \IEEEpubidadjcol in the second
% column for its text to clear the IEEEpubid mark (Computer Society jorunal
% papers don't need this extra clearance.)

% for Computer Society papers, we must declare the abstract and index terms
% PRIOR to the title within the \IEEEcompsoctitleabstractindextext IEEEtran
% command as these need to go into the title area created by \maketitle.
\IEEEcompsoctitleabstractindextext{%
\begin{abstract}
%\boldmath
A number of studies have reported the phenomenon of ``software aging'',
characterized by progressive software performance degradation.
Response time (RT) as a customer-affecting metric can be used to detect
the onset of software aging.
In this paper, we propose the distribution based rejuvenation algorithm
(DBRA)by using a validated M/E$_2$/1/K queueing model of Apache web server
that is used to help detect aging.
We compare the DBRA  with the
SRAA (static software rejuvenation with averaging) presented by Alberto Avritzer et.al.
Simulation results show the effectiveness of DBRA and is advantage over SRAA
to help set various control parameters.
\end{abstract}
% IEEEtran.cls defaults to using nonbold math in the Abstract.
% This preserves the distinction between vectors and scalars. However,
% if the journal you are submitting to favors bold math in the abstract,
% then you can use LaTeX's standard command \boldmath at the very start
% of the abstract to achieve this. Many IEEE journals frown on math
% in the abstract anyway. In particular, the Computer Society does
% not want either math or citations to appear in the abstract.

% Note that keywords are not normally used for peer review papers.
\begin{keywords}
queueing model, response time distribution, distribution based rejuvenation algorithm,
static software rejuvenation with averaging, software aging detection
\end{keywords}}

% make the title area
\maketitle

% To allow for easy dual compilation without having to reenter the
% abstract/keywords data, the \IEEEcompsoctitleabstractindextext text will
% not be used in maketitle, but will appear (i.e., to be "transported")
% here as \IEEEdisplaynotcompsoctitleabstractindextext when compsoc mode
% is not selected <OR> if conference mode is selected - because compsoc
% conference papers position the abstract like regular (non-compsoc)
% papers do!
\IEEEdisplaynotcompsoctitleabstractindextext
% \IEEEdisplaynotcompsoctitleabstractindextext has no effect when using
% compsoc under a non-conference mode.


% For peer review papers, you can put extra information on the cover
% page as needed:
% \ifCLASSOPTIONpeerreview
% \begin{center} \bfseries EDICS Category: 3-BBND \end{center}
% \fi
%
% For peerreview papers, this IEEEtran command inserts a page break and
% creates the second title. It will be ignored for other modes.
\IEEEpeerreviewmaketitle

\section{Introduction} % I
\label{intro}
It is well known that system outages are more due to software faults than due to hardware faults \cite{ChScience:SGarg1998IEEEJC}.
Software system faults can be classified into three types according to the
potential manifestation characteristic:
Bohrbugs, Mandelbugs, and Aging-related bugs,
a sub-type of the Mandelbugs \cite{ChScience:MGrottke2007IEEEMC},
Grottke et al. analyzed the faults discovered in the on-board software for
18 JPL/NASA space missions \cite{ChScience:MGrottke2010DSN}
based on this classification method. Software aging is the phenomenon of
progressive performance degradation of the running software,
which may lead to system crashes or undesirable hangs \cite{ChScience:SGarg1998ISSRE}.
It may happen due to the exhaustion of system resources, such as memory-leaks,
unreleased locks, non-terminated threads, shared-memory pool latching,
storage fragmentation, or the like \cite{ChScience:YHuang1995SFTC}.
This undesired phenomenon exists not only in generally used software such as web
and application servers, but also in critical applications requiring high
reliability/availability.
Software aging could cause great losses in
the safety-critical systems \cite{ChScience:YFJia2008APSEC},
including the loss of human lives \cite{ChScience:EMarshall1992Science}.
Software aging does not make software fail immediately once started,
but instead it leads to progressive performance degradation of the system until
the system hangs or crashes.
To counteract software aging, researchers have proposed a proactive approach
called software rejuvenation (SR) \cite{ChScience:SGarg1998ISSRE}.
Rejuvenation has been implemented in various systems,
such as billing data collection systems, telecommunication systems,
transaction processing systems and spacecraft
systems \cite{ChScience:KYCai2006JCST}, \cite{ChScience:KJCassidy2002ICDS},
\cite{ChScience:XMZhang2002JSS}.
It involves occasionally terminating the system,
cleaning its internal state and restarting it in order to release system resources,
so that the software performance is recovered.
Thus, SR mends the system before it fails.

The aging behavior can be captured by one or more indicators of aging.
Such indicators are measurable metrics of the server likely to be influenced
by software aging.
Software aging and performance degradation can be gauged by monitoring system
resources from or monitoring at the application level.
Measurable metrics of system resources, are amount of memory free,
swap space free, etc., while response time is a key measure of the performance
at the user/application level.
The variation of response time can be used to infer the evolving process of
software aging.
From the client’s perspective, gradually increasing response time can be
a sign of software aging caused by performance degradation of the server.
Response time (RT) as a customer-affecting metric obtained by continuous monitoring,
can be used to detect the need for rejuvenation so as to counteract the effect
of software performance degradation. 
Alberto Avritzer et al. have wittnessed the aging in telecomcommunations'
software, where the service rate of the software decreases with time
increasing the queue lengths and eventually starts losing packets.
By frequently monitoring the metric of RT,
the average of RT samples can be obtained.
Alberto Avritzer et al. built an analytical model of
the M/M/c queueing model, and
proposed algorithms for the detection software aging
and subsequent rejuvenation
based on the measured RT values \cite{ChScience:AAvritzer2006DSN}.

The online monitoring algorithm can distinguish between the system performance
degradation that is a consequence of aging and that due to bursts
in the arrival process.
By frequently sampling the customer affecting metric based on actual
measurements and by storing the measurement results that are larger than
the expected value of the customer affecting metric as a continuous set of buckets,
a ball is removed or added to the current bucket.
This process is repeated until it is monitored that the customer affecting metric
degrades so much to become larger than the expected value plus the number of
standard deviations \cite{ChScience:AAvritzer2006DSN}.
Alberto Avritzer et al. proposed three algorithms,and used the number of buckets,
the sample size of RT, and the bucket depth 
to control the time to rejuvenation.
In the first algorithm, called static software rejuvenation with averaging (SRAA), 
the observed values of RT are averaged over a fixed sample size.
The second algorithm, called static software rejuvenation with averaging 
and sample acceleration (SARAA), reduces the sample size when a degradation in 
performance is detected. The two algorithms smooth short-term deviations
of the RT by using multiple buckets, the bucket depth, and the number of 
successive observations. Compared with the above mentioned algorithms using 
small sample size, the third algorithm called CLTA use the large sample size 
to approximate the normal distribution, and then follow the central limit 
theorem. The performance of SRAA, SARAA and CLTA algorithm is evaluated 
by adjusting the control parameters using simulation. 
We collected the average of the measured response time samples and
when it exceeds the target quantile of the RT distribution,
a rejuvenation is then triggered.
To obtain the target quantile of RT distribution,
an analytic model is developed here and utilized.
Based on the distribution of mean of RT, we develop the distribution based
rejuvenation algorithm (DBRA) to directly control the rejuvenation, instead of using the static
software rejuvenation with averaging algorithm(SRAA) 
presented by Avritzer.

The Apache server has been the most popular web server used on
the Internet \cite{ChScience:Apache},
and is known to suffer from software aging \cite{ChScience:MGrottke2006IEEEJR}.
Since the Apache web server, the most commonly used web server to
date \cite{ChScience:NetcraftWSSA} \cite{ChScience:MGrottke2006IEEEJR},
we select it conduct software aging and rejuvenation experiments.
The sample average of measured RT is compared against quantile of
theoretical response time to determine if measured RT significantly exceeds
the nominal value that suggesting onset or aging.
While the Apache server is multi-threaded,
a request is handled by its own thread or process throughout
the life cycle of the request, a limit is put on the number of processes
allowed at one time in the server, and some parameters need to be
customized \cite{ChScience:SGarg1998IEEEJC}, \cite{ChScience:YFJia2008ISSRE},
\cite{ChScience:JCao2003ICT}.
The effect of aging of the Apache server may result in unavailability
as well as the gradual performance degradation,
or the gradually decreasing service rate \cite{ChScience:SGarg1998IEEEJC}.
In practice, these kinds of factors need to be considered to build the
queueing model of the Apache web server.
Therefore, an accurate analytical model of the Apache web server is needed
to be built and validated by experimental measurement,
and then the response time distribution can be obtained,
which may facilitate in triggering rejuvenation using the algorithm of DBRA
mentioned above.

In summary, our main contributions are as follows:
\begin{itemize}
 \item We propose the distribution based algorithm to ensure the performance of Apache web server.
 It use the degree of confidence directly to control when to rejuvenation,
 whereas the bucket algorithm such as SRAA, employ the number of buckets and 
 bucket depth to control the rejuvenation.
 It has some advantages over the bucket algorithm proposed by Avrizter.    
 \item We first validate the Apache web server at the aging setting 
 using M/E$_2$/1/K queuing model. Furthermore, we calculate the RT distribution, 
 and obtain the numerical value of RT distribution.
 \item We combine the analytical model, the simulation model and experiment together to ensure the performance of Apache web server.
\end{itemize}

This paper is an extension of our previous conference paper\cite{ChScience:JingZhaoWoSAR11}. 
We have made the following new contributions. First, we propose the distribution base rejuvenation algorithm(DBRA) 
using the degree of confidence to control the rejuvenation. 
Second, we numerically obtain the quantile for the specific degree of confidence
using SHARPE \cite{ChScience:KSTrivedi2005ACM}.
Third, we develop the simulation program to validate the distribution based rejuvenation algorithm(DBRA),
and also we compare the effectiveness of DBRA and SRAA with changing control parameters.

The rest of the paper is organized as follows.
In Section \ref{sec:relatedwork}, we 
present an overview of related work.
In Section \ref{sec:queueingModel},
the queueing model of the Apache web server is built based on Markov chains,
and is validated by experimental results.
In Section \ref{sec:responseTimeDist},
firstly, RT distribution of M/E$_2$/1/K model, $F_R(t)$,
is obtained by tagged job approach.
The mean and variance of $F_R(t)$ are derived by solving approximation to
the absorption time distribution of a continuous-time Markov chain (CTMC).
Secondly, distribution of the mean of RT, $F_{\overline{R}}$,
is calculated by time to absorption of CTMC,
the mean and variance of $\overline{R_n}$, is also calculated.
We numerically calculated using SHARPE \cite{ChScience:KSTrivedi2005ACM}.
In Section \ref{sec:simResults},
the distribution based rejuvenation algorithm (DBRA) is proposed,
and we compare the performance of DBRA with the SRAA presented by Alberto by simulation.
The effectiveness of these two algorithms are compared by
adjusting the control parameters.
Finally, concluding remarks and future research are discussed
in Section \ref{sec:conclusion}.

\section{Related Work}
\label{sec:relatedwork}
To the best of our knowledge, there exists two other approaches
in several related areas. Admission and flow control algorithms 
are designed to prevent from overload and speed mismatches, and 
thus guarantee the required quality of service (QoS) of the network. 
Flow control algorithms are designed to balance the load to protect 
the bottleneck resources. Network flow control mechanisms has been 
classified into hop-level flow control, entry-to-exit flow control,
and network access flow control \cite{ChScience:AAvritzer1997ESE}.
Gilly et al. proposed an adaptive control and load balancing algorithm
for the web system. They proposed an adaptive time slot
to set the frequencies of the algorithm invoked, in order to achieve
the low overhead of the algorithm \cite{Gilly2012informationscience}.
Flow control supports high quality real-time applications such as voice 
and video, et al. presents the application flow control technique
used by YouTube \cite{ChScience:SAlcock2011ACMSIGCOMMCCR}.
Menth et al. give a survey of pre-congestion notification based admission
control method in an early stage of standardized process \cite{ChScience:Michael2010IEEEcommunition}.
Chen et al. proposed a utility based flow control algorithm for 
wireless sensor network, they formulate a flow control optimization problem
with life time constraint and link interference in an asynchronous 
setting \cite{ChScience:JChen2010IEEEJournal}
%%\cite

Another related area is the statistical process control(SPC) method for quality 
management. SPC use statistical methods and procedures (such as control charts)
to analyze the inherent variability of a process to achieve and maintain a state 
of statistical control, and to improve the process capability.
SPC has been widely used in the manufacturing industry for a long time.
Control charts has two assumptions that hard to meet the software system performance
testing, one assumption is 
that the outputs have a uni-modal normal distribution,
and the second assumption is that the load inputs do not vary across runs.
In order to apply the SPC method to software system 
performance regression testing, Nguyen et al. 
proposed an approach to customize the control charts to automatically detect the 
performance regressions \cite{ChScience:THDNguyen2012ACM}.

\section{The Queueing Model of Apache Web Server} % II
\label{sec:queueingModel}
The queueing model of the Apache web server is built as follows:
Firstly, the experimental setting of the Apache web server is introduced
in detail;
secondly, the queueing model of the Apache web server is built based
on continuous time Markov chain (CTMC),
the mean of RT obtained from the queueing model is validated by
the measurement RT.

\subsection{Apache web server} % 2.1
\label{subsec:apacheWebServer}
The Apache web server is structured as a pool of workers
(either threads or processes, depending on the specific software release),
as shown in Figure \ref{fig:apacheArch}.

\begin{figure}[htb]
\centering
\includegraphics[width=0.45\textwidth]{apacheArchitecuture}
\caption{Apache architecture}
\label{fig:apacheArch}
\end{figure}

Requests enter the server at the accept queue,
where they wait until a worker is available.
A worker is available if it is in the ``idle'' state.
While the worker is processing a request, it is in the ``busy'' state.
The widely used HTTP 1.1 protocol provides for persistent connections.
The Maxclients parameter limits the size of this worker pool,
thereby imposing a limitation on the processing rate of the server.
A parent process is responsible for launching child processes to
handle requests, and adjusting the child processes by killing
or spawn to meet workload.

The capacity of the web server depends on its configuration.
To assess the impact of MaxRequestsPerChild and MaxClients,
we adjust the parameter that can speed up the accumulation of side effects
of residual defects, i.e, MaxResquestPerchild.
This parameter is set to zero, which means no child processes will
ever be killed.
The other parameter, i.e, MaxClients is set to 250.
The connection rate of the data set varies from 350 per second to 390
per second during a 14-day period to represent different workload levels.
We conclude that under this setting the capacity of the web server is
about 390 requests per second.
Data were recorded at five-minute intervals.

\subsection{Queueing model for the Apache web server} % 2.2
\label{subsec:queueingModel}
The queueing model is built according to the Apache working mechanism as
well as experimental parameter settings.
Transaction service rate, $\mu(t)$ is defined to be a monotone
non-increasing function of $t$ \cite{ChScience:SGarg1998IEEEJC},
and this behavior of $\mu(t)$ was used as an approximation to service
degradation in telecommunications
switching software \cite{ChScience:AAvritzer1997ESE}.
The constant service rate of the exponential distribution can clearly
not represent gradually decreasing rate,
whereas the Erlang distribution provides much more flexibility
than the exponential distribution,
and is a reasonable assumption for representing the software aging.
Even though the transaction service may not actually consist of $r$ phases,
it still can be regarded as having the Erlang distribution.
Thus, the service time is denoted by the $r$-stage Erlang distribution,
service discipline is FCFS,
arriving transactions are queued and the buffer size is $K$.
The queueing model is thus M/E$_r$/1/K \cite{ChScience:KSTrivedi2001PS},
which can be represented by the CTMC shown in Figure \ref{fig:MME2K}.
In this paper, we select a $2$-stage Erlang as service time distribution,
thus the queueing model is M/E$_2$/1/K.

\begin{figure}[htb]
\centering
\includegraphics[width=0.5\textwidth]{MME2K}
\caption{M/E$_2$/1/k transform state}
\label{fig:MME2K}
\end{figure}

The steady-state balance equations of the M/E$_r$/1 model can be
written as follows \cite{ChScience:DGross1985FQ},
where $n$ is the number of phases in the system:
\begin{eqnarray}
\label{equ:eq1}
  \begin{cases}
    0=-(\lambda+k\mu)\pi_n+k\mu \pi_{n+1}+\lambda \pi_{n-k} & n\geq 1 \cr
    0=-\lambda \pi_0+k\mu \pi_1 &
  \end{cases}
\end{eqnarray}

We can rewrite the M/E$_2$/1/K model into the following
steady-state balance equations,
\begin{eqnarray}
\label{equ:eq2}
  \begin{cases}
    0=-(\lambda+2\mu)p_n+2\mu p_{n+1}+\lambda p_{n-2}\cr
            ~~~~~~~~(1\leq n\leq 2K-2) \cr
    0=-\lambda p_0+2\mu p_1 (n=0) \cr
    0=-2\mu p_{2k-1}+2\mu p_{2k}+\lambda p_{2k-3}(n=2K-1) \cr
    0=-2\mu p_{2k}+\lambda p_{2k-2}(n=2K) \cr
    \sum_{n=0}^{2k}p_n=1
  \end{cases}
\end{eqnarray}

The steady state $\pi_n$ of M/E$_2$/1 can be simplified to
solve the Bulk Input model of $M^{[x]}$/M/1.

With,
\begin{equation}
\label{equ:eq3}
p_0=1-\frac{\lambda}{\mu}=1-\rho
\end{equation}

We use a generating function approach, where
\begin{equation}
\label{equ:eq4}
P(z)=\sum_{n=0}^{\infty} p_nz^n(\left\vert z\right\vert\leq 1),~and~C(z)=\sum_{n=1}^{\infty} c_nz^n(\left\vert z\right\vert\leq 1)
\end{equation}

Making the above transformations in (\ref{equ:eq1}) yields,
\begin{equation}
\label{equ:eq5}
P(z)=\frac{\mu^{(B)}p_0(1-z)}{\mu^{(B)}(1-z)-\lambda z(1-C(z))}(\left\vert z\right\vert\leq 1)
\end{equation}
where, $\mu^{(B)}=B\mu$

We can use formula (\ref{equ:eq5}) directly to obtain $P(z)$,
realizing that $C(z)$ is merely equal to $z^B$,
since the batch size is fixed at $B$ equaling 2.
We can obtain here
\begin{equation}
\label{equ:eq6}
P(z)=\frac{2\mu p_0(1-z)}{2\mu(1-z)-\lambda z(1-z^2)}(\left\vert z\right\vert\leq 1)
\end{equation}

Further, formula (\ref{equ:eq6}) can be simplified to
\begin{equation}
\label{equ:eq7}
P(z)=-\frac{2}{\rho}\cdot\frac{1-\rho}{z^2+z-2/\rho}(\left\vert z\right\vert\leq 1)
\end{equation}

Partial fraction of formula (\ref{equ:eq7}) yields,
\begin{eqnarray}
\label{equ:eq8}
  \begin{aligned}
  &P(z)=\frac{2}{\rho}\cdot \frac{1-\rho}{\sqrt{1+8/\rho}}\cdot \cr
  &\left[
    \begin{aligned}
      &\frac{2}{(-1+\sqrt{1+8/\rho})}\cdot\frac{1}{1-\frac{2}{-1+\sqrt{1+8/\rho}}\cdot z}-\cr
      &\frac{2}{(-1-\sqrt{1+8/\rho})}\cdot\frac{1}{1-\frac{2}{-1-\sqrt{1+8/\rho}}\cdot z}
    \end{aligned}
  \right] \cr
  &~~~~~~~~(\left\vert z\right\vert\leq 1)
  \end{aligned}
\end{eqnarray}

Formula \ref{equ:eq8} can be written in Taylor series formula as follows:
\begin{eqnarray}
\label{equ:eq9}
  \begin{aligned}
  &P(z)=\frac{2}{\rho}\cdot \frac{1-\rho}{\sqrt{1+8/\rho}}\cdot\sum_{n=0}^{\infty}\left( (\frac{2}{-1+\sqrt{1+8/\rho}})^{n+1}\right. \cr
  &\left. -(\frac{2}{-1-\sqrt{1+8/\rho}})^{n+1} \right)\cdot z^n(\left\vert z\right\vert\leq 1)
  \end{aligned}
\end{eqnarray}

$\pi_n$, the probability of $n$ phases in the Erlang service system,
is given by
\begin{eqnarray}
\label{equ:eq10}
  \begin{aligned}
  &\pi_n=\frac{2}{\rho}\cdot \frac{1-\rho}{\sqrt{1+8/\rho}}\cdot \left( (\frac{2}{-1+\sqrt{1+8/\rho}})^{n+1}\right. \cr
  &\left. -(\frac{2}{-1-\sqrt{1+8/\rho}})^{n+1} \right)
  \end{aligned}
\end{eqnarray}

The steady state $p_n$ of M/E$_2$/1/K is normalized by
\begin{equation}
\label{equ:eq11}
p_n=\frac{1}{C}\pi_n
\end{equation}

Summing two sides of (\ref{equ:eq11}), we derive
\begin{equation}
\label{equ:eq12}
\sum_{n=0}^{2K}p_n=\frac{1}{C}\sum_{n=0}^{2K}\pi_n=1
\end{equation}

We obtain $C$ from (\ref{equ:eq12}),
let $q_0=\frac{2}{-1+\sqrt{1+8/\rho}}$, $q_1=\frac{2}{-1+\sqrt{1+8/\rho}}$,
\begin{equation}
\label{equ:eq13}
C=\frac{2}{\rho}\cdot\frac{1-\rho}{\sqrt{1+8/\rho}}\left( \frac{q_0-q_0^{2K+2}}{1-q_0}-\frac{q_1-q_1^{2K+2}}{1-q_1} \right)
\end{equation}

Thus,
\begin{equation}
\label{equ:eq14}
p_n=\frac{1}{C}\pi_n=\frac{1}{C}\cdot\frac{2}{\rho}\cdot\frac{1-\rho}{\sqrt{1+8/\rho}}\cdot[q_0^{n+1}-q_1^{n+1}]
\end{equation}

The probability of blocking due to buffer full, $P_b=P_{2K}$,
and $K=250$.
The probability of acceptance of the M/E$_2$/1/K model, $P_a$, is given by
\begin{equation}
\label{equ:eq15}
P_a=1-p_{2K}-p_{2K-1}
\end{equation}

Let $E[R]$ represents the mean of the RT,
the sojourn time in the queue at steady state plus the service time,
we can then obtain
\begin{equation}
\label{equ:eq16}
E[R]=\frac{\sum_{n=0}^{K*2-2}np_n}{2\mu}+\frac{1}{\mu}
\end{equation}

We quantify the mean values of the response time, $E[R]$,
which is validated by actual measured data of RT.
The comparison results among the model outputs, the average measurements of RT,
and the RT with the queue model of M/M/1/K,
are shown in Table \ref{table:tab1}.
As for the queue model of M/M/1/K, 
the inter-arrival time is exponential, and time-to service observe
the exponential distribution.
Furthermore, the comprison among the model outputs when $r$ equals to 2, 3, 5, and 10, 
for the M/E$_r$/1/250 model, are shown in Table \ref{table:tabr}, respectively.
From tables \ref{table:tab1} and \ref{table:tabr},
we can see that M/E$_2$/1/K model is appropriate to characterize
the actual measurement.

\begin{table*}[!t]
\renewcommand{\arraystretch}{1.3}
\caption{Comparison results of RT among measurements, M/E$_2$/1/K and M/M/1/K }
\label{table:tab1}
\centering
\begin{tabular}{ c c c c c }
\hline
$\lambda$    & $\mu$        & measurement of        & RT of         & RT of     \\
(reqs/s)     & (reqs/s)     & RT                & M/E$_2$/1/250 & M/M/1/250 \\
             &              & (ms)              & (ms)          & (ms)      \\
\hline
350          & 390          & 20.2449           & 19.39         & 25        \\
360          & 390          & 29.0386           & 25.64         & 33        \\
370          & 390          & 38.8456           & 38.15         & 50        \\
380          & 390          & 103.2922          & 75.5          & 99        \\
\hline
\end{tabular}
\end{table*}


\begin{table*}
\renewcommand{\arraystretch}{1.3}
\caption{RT of M/E$_r$/1/K when r equals to 3, 4, 5, and 10}
\label{table:tabr}
\centering
\begin{tabular}{ c c c c c c c }
\hline
$\lambda$     & $\mu$     & RT of         & RT of             & RT of         & RT of         & RT of     \\
(reqs/s)      & (reqs/s)  & M/E$_2$/1/250 & M/E$_3$/1/250     & M/E$_4$/1/250 & M/E$_5$/1/250 & M/E$_{10}$/1/250 \\
              &           & (ms)          & (ms)              & (ms)          & (ms)          & (ms)      \\
\hline                                                                                                   
350           & 390       & 19.39         & 16.67           & 15.60         & 14.99         & 13.75        \\
360           & 390       & 25.64         & 22.20           & 20.84         & 20.01         & 18.33        \\
370           & 390       & 38.15         & 33.30           & 31.25         & 30.00         & 27.49        \\
380           & 390       & 75.5          & 68.55          & 63.80          & 60.90          & 54.99        \\
\hline
\end{tabular}
\end{table*}

\section{Response time distribution} % III
\label{sec:responseTimeDist}
Woolet presented a response time computation technique,
and the key idea works as follows:
Assume that the response time distribution is of phase-type, i.e.,
it corresponds to the absorption time distribution of
a continuous-time Markov chain (CTMC) \cite{ChScience:SPWoolet1993PHD}.

Let $\overline{R_n}$ denote the average
$\overline{R_n}=\frac{1}{n}\sum_{m=1}^nR_m$ of $n$ independent
and identically distributed random variables $R_m$,
we can derive the probability density function of $\overline{R_n}$,
mean and variance of $\overline{R_n}$.
We can then base the decision of whether to trigger rejuvenation on
the quantiles of the distribution of $\overline{R_n}$.

In this section,
we first calculate the RT distribution, $F_R(t)$,
and then calculate the mean and variance of $F_R(t)$.
Secondly, we also calculate the PDF (probability density function),
mean and variance of $\overline{R_n}$.

\subsection{CDF, mean and variance of R} % 3.1
\label{subsec:CDFofR}
We use tagged job approach to calculate the response time distribution of
R of M/E$_2$/1/K \cite{ChScience:KSTrivedi2001PS},
which can be denoted as the formula as follows:
\begin{equation}
\label{equ:eq17}
F_R(t)=\sum_{i=0}^{2K-2}p_i\cdot \left[ 1-\sum_{j=0}^{i+1}\frac{(2\mu t)^je^{2\mu t}}{j!} \right]
\end{equation}

The response time distribution of R can be denoted as the absorption time
distribution of a continuous-time Markov chain (CTMC), the state diagram
of the CTMC is given in Figure \ref{fig:responseTimeDist}.

\begin{figure*}[htb]
\centering
\includegraphics[width=0.9\textwidth]{responseTimeDist}
\caption{The state diagram of time to absorption of R}
\label{fig:responseTimeDist}
\end{figure*}

Differential equations for this CTMC yields
\begin{equation}
\label{equ:eq18}
  \begin{cases}
    \frac{\du\pi_{T_1}(t)}{\du t}=-2\mu\cdot\pi_{T_1}(t)~~(i=1) \\
    \frac{\du\pi_{T_2}(t)}{\du t}=-2\mu\cdot\pi_{T_2}(t)+2\mu\cdot\pi_{T_1}(t)~~(i=2) \\
    \frac{\du\pi_{T_i}(t)}{\du t}=-2\mu\cdot\pi_{T_i}(t)+2\mu\cdot(1-q_{i-3})\cdot\pi_{T_{i-1}}(t) \\
    ~~~~~~~~(3\leq i\leq 2K) \\
    \frac{\du\pi_{T_R}(t)}{\du t}=\sum_{i=2}^{2K}[2\mu\cdot q_{i-2}\cdot \pi_R(t)]
  \end{cases}
\end{equation}
where we have assumed that the initial state is $T_1$
so that $\pi_{T_1}(0)=1$, and $\pi_{T_i}(0)=0=\pi_{T_R}(0)$,
using the Laplace transform, we derive
\begin{equation}
\label{equ:eq19}
s\overline{\pi_R(s)}=\sum_{i=2}^{2K}2\mu q_{i-2}\pi_{T_i}(s)
\end{equation}
where,
\begin{equation}
\label{equ:eq20}
  \begin{split}
    q_n=&\frac{p_n}{p_n+p_{n+1}+\ldots+p_{2k-2}} \\
       =&\frac{q_0^{n+1}-q_1^{n+1}}{\frac{q_0^{n+1}-q_0^{2K}}{1-q_0}-\frac{q_1^{n+1}-q_1^{2K}}{1-q_1}}
  \end{split}
\end{equation}

By solving formula (\ref{equ:eq19}) and (\ref{equ:eq20}), we obtain
\begin{equation}
\label{equ:eq21}
s\overline{\pi_R(s)}=\sum_{i=2}^{2K}2\mu q_{i-2}\pi_{T_i}(s)=\sum_{i=0}^{2K-2}P_i\frac{(2\mu)^{i+2}}{(s+2\mu)^{i+2}}
\end{equation}

From formula (\ref{equ:eq22}), we obtain $E(R)$,
\begin{equation}
\label{equ:eq22}
E(R)=-\frac{\du\overline{f_R}(s)}{\du s}|_{s=0}=\sum_{i=0}^{2K-2}P_i\cdot\frac{i+1}{2\mu}
\end{equation}

We obtain $\sigma^2$, the variance of $R$,
\begin{equation}
\label{equ:eq23}
  \begin{split}
    \sigma^2=&E(R^2)-(E(R))^2 \\
            =&\sum_{i=0}^{2K-2}(i+1)(i+2)\frac{1}{4\mu^2}-\left[ \sum_{i=0}^{2K-2}p_i\cdot(i+1)\frac{1}{2\mu} \right]^2
  \end{split}
\end{equation}

\subsection{CDF, mean and variance of $\overline{R_n}$} % 3.2
\label{subsec:CDFofRn}
Let $\overline{R_n}=\sum_{m=1}^nR_m/n$,  
%%\begin{equation}
%%\label{equ:eq24}
%%\overline{R_n}=\sum_{m=1}^nY_m
%%\end{equation}
here, $R_m/n,~m=1, 2, \dots,n$ , are independent and identically distributed
random variables, and
each individual summand $R_m/n$ is the response time of a job in an
M/E$_2$/1/K. 
let $E(\overline{R_n})$ denote the expected value of $\overline{R_n}$,
thus, we can derive
\begin{equation}
\label{equ:eq25}
E(\overline{R_n})=\frac{1}{n}\cdot n\cdot E(R_m)=E(R)
\end{equation}

Let $Var(\overline{R_n})$ denote the variance of $\overline{R_n}$,
$\sigma^2$ denote the variance of $R_m$, we can derive
\begin{equation}
\label{equ:eq26}
Var(\overline{R_n})=Var\left( \frac{R_1+R_2+\ldots+R_n}{n} \right)=\frac{\sigma^2}{n}
\end{equation}

To find the time to absorption of $\overline{R_n}$,
we begin by multiplying all transition rates
in Figure \ref{fig:responseTimeDist} by $n$ and get a Markov chain
in which the distribution of the time to absorption is identical to
the distribution of each individual $R_m/n$.
Because all transition times in a Markov chain are exponential and 
because dividing an exponentially distributed random variable with
hazard rate $z$ by some constant $r$ yields an exponentially distributed
random variable with hazard rate $r \cdot z$.
Therefore, $\overline{R_n}=\sum_{m=1}^nY_m$ can be represented as time
to absorption in the Markov chain derived by concatenating $n$ such
Markov chains, fusing state $F_m$ of the $m$th sub-chain and state
$T_{1,m+1}$ of the $(m+1)$st sub-chain into state $F_n$
of the resulting Markov chain.
This is shown in Figure \ref{fig:meanOfRT}.
We use SHARPE numerically calculate the RT distribution of $\overline{R_n}$
when $n$ equals 1, 5, 10, 15 and $\rho=$350/390 respectively,
as is shown in Figure \ref{fig:RTdist}.

\begin{figure}[htb]
\centering
\includegraphics[width=0.45\textwidth]{meanOfRT}
\caption{The state diagram of time to absorption of $\overline{R_n}$}
\label{fig:meanOfRT}
\end{figure}

\begin{figure}[htb]
\centering
\includegraphics[width=0.45\textwidth]{rtProb}
\caption{RT distribution of $\overline{R_n}$ at different $n$}
\label{fig:RTdist}
\end{figure}

\section{Simulation results for rejuvenation algorithms} % IV
\label{sec:simResults}
Based on the analytical results discussed
in section \ref{sec:responseTimeDist},
we first employ the SRAA algorithm presented by Alberto Avritzer et.al.
Secondly, we develop the distributed based rejuvenation algorithm(DBRA).
Finally, we compare the DBRA with the SRAA.


we employ the SRAA developed by Avritzer et.al,
the algorithm first track the statistical deviation of RT,
and then assure the performance measured by RT via
software rejuvenation \cite{ChScience:AAvritzer2006DSN}.
The algorithm called static software rejuvenation with averaging
(SRAA), where the observed values of RT are averaged over a fixed
sample size.
SRAA tries to verify that the distribution of RT
has shifted by a specific amount.

We evaluate the rejuvenation algorithm SRAA and DBRA
by running the simulation model of queueing model M/E$_2$/1/K.
We ran each simulation for 2500,000 transactions divided into
five replications of 500,000 transactions each.
In each simulation experiment, the value of $\mu_X$ and $\sigma_X$
when  $\rho=$350/390, 360/390, 370/390, and 380/390 are used respectively.
Average RT and fractions of transactions lost are shown as functions
of the offered load ($\rho=\lambda/\mu$).
We kept the value of the product of sample size, $n$, number of buckets,
$K$, and bucket depth, $D$, constant,
and refer to this as $n\cdot K\cdot D$.

\subsection{SRAA, $n\cdot K\cdot D=300$} % 4.1
\label{subsec:SRAAnKD300}
Figure \ref{fig:SRAA:a} presents the RT results for SRAA,
with $n\cdot K\cdot D=300$.
We performed experiments setting
$(n, K, D)=(1, 10, 30), (2, 10, 15), (3, 10, 10), (6, 10, 5)$.
We do not see a clear dichotomy of the RTs for these four settings.
However, when we examine Figure \ref{fig:SRAA:b},
we see clearly that the average fraction loss is different,
when $(n, K, D)=(2, 10, 15)$, it incurs a higher transaction loss, while
when $(n, K, D)=(6, 10, 5)$, the transaction loss is lowest at both
lower load and higher load.
Here, we observe the sample size $n$ has a significant effect on
the average transaction loss, since frequently sampled RT will incur
software rejuvenation times more than that of infrequently sampled RT.

\begin{figure}[htb]
\centering
\subfigure[Response time results, SRAA, $n\cdot K\cdot D=300$]{
           \includegraphics[width=0.45\textwidth]{SRAAa}
           \label{fig:SRAA:a}
          }
%\hfill
\\
\subfigure[Fraction of transaction loss results, SRAA, $n\cdot K\cdot D=300$]{
           \includegraphics[width=0.45\textwidth]{SRAAb}
           \label{fig:SRAA:b}
          }
\caption{Response time and transaction loss results when $n\cdot K\cdot D=300$}
\label{fig:SRAA}
\end{figure}

Also, we assume the rejuvenation time is 2 seconds in our simulation model,
from simulation results we obtain that the rejuvenation times
for setting $(6, 10, 5)$ is from 60 times to 110 times,
which is least compared with other three cases.
From Figure \ref{fig:SRAA:a} we can see the average fraction loss
decrease when $\rho=380/390$, since the analytical average RT,
$\mu_X$ used in our rejuvenation algorithm, is 76.92ms,
while the simulation average RT without rejuvenation is 65.38ms,
so that we can see this decreasing average
fraction loss when $\rho=380/390$.

\begin{figure}[htb]
\centering
\subfigure[RT when sample size doubled]{
           \includegraphics[width=0.45\textwidth]{sizeDoubleda}
           \label{fig:sizeDoubled:a}
          }
%\hfil
\\
\subfigure[Average Fraction Loss when sample size doubled]{
           \includegraphics[width=0.45\textwidth]{sizeDoubledb}
           \label{fig:sizeDoubled:b}
          }
\caption{Response time and transaction loss results when $n\cdot K\cdot D=600$}
\label{fig:sizeDoubled}
\end{figure}

\subsection{SRAA, sample size doubled} % 4.2
\label{subsec:SRAAsampleDoubled}
Figure \ref{fig:sizeDoubled:a} presents RT results for SRAA,
with $n\cdot K\cdot D=600$.
We set $(n, K, D)=(2, 10, 30), (4, 10, 15), (6, 10, 10), (12, 10, 5)$,
by doubling the sample size of the experimental settings
from part \ref{subsec:SRAAnKD300} of section \ref{sec:simResults}.
From Figure \ref{fig:sizeDoubled:a} we see the similar RT results
for these four settings, while from Figure \ref{fig:sizeDoubled:b},
we see clearly average transaction loss different for these four settings,
we can see, $(n, K, D)=(4, 10, 15)$ incur a higher transaction loss
when the sample size is 2, while when $(n, K, D)=(12, 10, 5)$,
the average transaction loss is lowest.

Doubling the values of sample size has a negative impact on RT.
For example, when $\rho=350/390$ and $(n, K, D)=(1, 10, 30)$,
the average RT for SRAA is 18.16388ms, while for $(n, K, D)=(2, 10, 30)$,
the average RT is 19.2835ms;
when $\rho=380/390$ and $(n, K, D)=(1, 10, 30)$,
the average RT for SRAA is 42.45974ms, while for $(n, K, D)=(2, 10, 30)$,
the average RT is 47.878ms.
The observed RTs are longer because the lager sample size means the
rejuvenation is triggered later,
as it takes longer to collect a larger sample.

\subsection{Comparing DBRA, SRAA } % 4.3
\label{subsec:comparing}
We develop the distribution based rejuvenation algorithm(DBRA) as shown in Algorithm \ref{alg:DBRA}.
It directly employ the mean of RT distribution to control the software rejuvenation
instead of trying the bucket deviation using the bucket depth, the bucket number,
and the observed sample size as the control parameters.
The algorithm works as follows:
TM denote the target value when the designed response time distribution of
$\overline{R_n}$ confidence interval reached.

\begin{algorithm}
%\begin{figure}[!h]
\caption{Distribution based rejuvenation algorithm}
\label{alg:DBRA}
\begin{algorithmic}[1]
\Function{DBRA}{d, n, TM}
    \State $u\gets 0$
    \State $d\gets 0$
    \While{n additional observations available}
        \State $u\gets u+1$
        \State $\overline{X_u}\gets \frac{1}{n}\sum^{un}_{(u-1)n+1}X_t$
        \If{$\overline{X_u}>TM$}
            \State $d\gets d+1$
        \Else
            \State $d\gets d-1$
        \EndIf
        \If{$d > D$}
            \State rejuvenation\_route()
            \State $d\gets 0$
        \EndIf
        \If{$d < 0$}
            \State $d\gets 0$;
        \EndIf
    \EndWhile
\EndFunction
\end{algorithmic}
%\end{figure}
\end{algorithm}

Figure \ref{fig:mRTd20:a} presents RT results for DBRA
with fixed depth $D=20$. We change the value of $n=1,5,10,15$, and
the parameter k is set implicitedly to 1.
Figure \ref{fig:TlossD20:b} presents the average transaction loss for these four
settings,
and it is clearly that the average transaction loss different for these four settings,
we can see, $n=5$  incur a higher transaction loss
, while when $n=15$,
the average transaction loss is lowest at the low arrival rate.
This can be explained that the larger sample size means the rejuvenation
trigger later, as it takes longer time to collect a larger sample,
so that the rejuvenation times is less compared with other settings.

\begin{figure}[htb]
\centering
\subfigure[Average RT when Depth fixed at $20$]{
           \includegraphics[width=0.45\textwidth]{mRTd=20}
          \label{fig:mRTd20:a}
          }
%\hfil
\\
\subfigure[Average Loss when Depth fixed at $20$]{
           \includegraphics[width=0.45\textwidth]{TlossD=20}
           \label{fig:TlossD20:b}
          }
\caption{Average RT and Loss for DBRA}
\end{figure}

Figure \ref{fig:mRTd50:a} presents RT results for DBRA with fixed depth $D=50$.
Figure \ref{fig:TlossD50:b} presents the average transaction loss for these four settings.
Figure \ref{fig:TlossD20:b} and Figure \ref{fig:TlossD50:b} give the average transaction
loss when bucket depth equals $D=20$ and $D=50$ respectively,
and it is clearly to see that the average transaction loss of $D=50$ is lower than
that of $D=20$, since the larger bucket depth can allow the larger deviation.
We compare the DBRA with SRAA, and we set the $n\cdot K\cdot D=300$,
From Figure \ref{fig:SRAA:a} and Figure \ref{fig:mRTd20:a},
we see that the DBRA and SRAA guarantee the similar average RT at different workload level,
From Figure \ref{fig:SRAA:b}we see that the lowest average transaction loss of SRAA of (k=10,D=5,n=6)
is 0.044, and the highest average transaction loss of SRAA of (k=10,D=5,n=6) is 0.086.
While from Figure \ref{fig:TlossD20:b} we see that the lowest average transaction loss of DBRA
is 0.031, and the highest of average transaction loss of DBRA is 0.062 at $n=15$.
Furthermore, we see clearly the average transaction loss of DBRA is much lower than that of SRAA
at different n value.

\begin{figure}[htb]
\centering
\subfigure[Average RT when Depth fixed at $50$]{
           \includegraphics[width=0.45\textwidth]{mRTd=50}
          \label{fig:mRTd50:a}
          }
%\hfil
\\
\subfigure[Average Loss when Depth fixed at $50$]{
           \includegraphics[width=0.45\textwidth]{TlossD=50}
           \label{fig:TlossD50:b}
          }
\caption{Average RT and Loss for DBRA at depth=50}
\end{figure}

We obtain the average RT and transaction loss if we fix n value and change D depth value.
Here we change D value from $20$ to $200$, and the n fixed value settings are $1$, $5$, and $10$
respectively.
Figure \ref{fig:mRTn1:a} and \ref{fig:Tlossn1:b} give the average RT and transaction loss
when n equals $1$.
Figure \ref{fig:mRTn5:a} and \ref{fig:Tlossn5:b} give the average RT and transaction loss
when we set $n=5$.
Figure \ref{fig:mRTn10:a} and \ref{fig:Tlossn10:b} give the average RT and transaction loss
when we set $n=10$ and the same D value shown in Figure \ref{fig:Tlossn5:b}.
It is clearly to see that the $n=10$ setting give the better performance than the $n=5$ setting
Since that the larger sample size shows the less rejuvenation times, and thus incur the smaller
transaction loss overhead.
From these Figures we see that the average transaction losses are lower than SRAA at the similar setting.

\begin{figure}[htb]
\centering
\subfigure[Average RT when sample size fixed at $1$]{
           \includegraphics[width=0.45\textwidth]{mRTn=1}
          \label{fig:mRTn1:a}
          }
%\hfil
\\
\subfigure[Average Loss when sample size fixed at $1$]{
           \includegraphics[width=0.45\textwidth]{Tlossn=1}
           \label{fig:Tlossn1:b}
          }
\caption{Average RT and Loss for DBRA  when $n=1$}
\end{figure}

\begin{figure}[htb]
\centering
\subfigure[Average RT when sample size fixed at $5$]{
           \includegraphics[width=0.45\textwidth]{mRTn=5}
          \label{fig:mRTn5:a}
          }
%\hfil
\\
\subfigure[Average Loss when sample size fixed at $5$]{
           \includegraphics[width=0.45\textwidth]{Tlossn=5}
           \label{fig:Tlossn5:b}
          }
\caption{Average RT and Loss for DBRA when  $n=5$}
\end{figure}

\begin{figure}[htb]
\centering
\subfigure[Average RT when sample size fixed at $10$]{
           \includegraphics[width=0.45\textwidth]{mRTn=10}
          \label{fig:mRTn10:a}
          }
%\hfil
\\
\subfigure[Average Loss when sample size fixed at $10$]{
           \includegraphics[width=0.45\textwidth]{Tlossn=10}
           \label{fig:Tlossn10:b}
          }
\caption{Average RT and Loss for DBRA at $n=10$}
\end{figure}

\section{Conclusion}
\label{sec:conclusion}
In this paper, a queueing model of M/E$_2$/1/K is built to analyze the
response time of the aging of the Apache web server,
and is validated by actual measured data.
Firstly, we obtain the RT distribution of of M/E$_2$/1/ K model,
$F_R(t)$, the mean and variance of $F_R(t)$.
Secondly, we obtain the distribution of the mean of RT,
$F_{\overline{R_n}}$,
the mean and variance of $\overline{R_n}$.
We also numerically calculated RT distribution of $\overline{R_n}$
using SHARPE.
Thirdly, we proposed the distribution based rejuvenation algorithm to
control the rejuvenation, and compare with the static software rejuvenation
with average algorithm(SRAA) presented by Alberto Avizer et.al.
Finally,the software rejuvenation algorithm of DBRA, and SRAA are
studied by simulation, and the performance of these two algorithms are
compared, results show that the performance of DBRA are
advantage over the SRAA.
We have found the optimum performance can be obtained by directly using
the defined confidence interval of RT distribution plus the control parameters of sample size,
or bucket depth.

% use section* for acknowledgement
\ifCLASSOPTIONcompsoc
  % The Computer Society usually uses the plural form
  \section*{Acknowledgments}
\else
  % regular IEEE prefers the singular form
  \section*{Acknowledgment}
\fi
This work is supported by the National Natural Science Foundation of
China under Grant No. 60873036,
the National Research Foundation for the Doctoral Program of
Higher Education of China No.20070217051,
the fundamental research funds for the Central Universities
(award number HEUCF100601, HEUCFT1007).
This research was also supported in part by the NASA Office of Safety
and Mission Assurance (OSMA) Software Assurance Research Program
(SARP) under a JPL subcontract \# 1440119.

% Can use something like this to put references on a page
% by themselves when using endfloat and the captionsoff option.
\ifCLASSOPTIONcaptionsoff
  \newpage
\fi

\bibliographystyle{IEEEtran}
\bibliography{IEEEabrv,references}

\end{document}
