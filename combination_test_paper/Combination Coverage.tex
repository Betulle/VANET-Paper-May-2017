% FILENAME:summary.tex
% AUTHOR:Yuliang
% Template of SUMMARY
% 2010-11-30
\documentclass{article}
\renewcommand{\baselinestretch}{2}
\usepackage{CJKutf8}
\usepackage{wrapfig}
\usepackage{graphicx}
\usepackage{hyperref}
\usepackage[margin=2cm]{geometry}

\hypersetup{
    colorlinks,
    citecolor=green,
    filecolor=magenta,
    linkcolor=red,
    urlcolor=blue
}
\usepackage{indentfirst}
\begin{document}
\large
\begin{CJK}{UTF8}{gkai}

\title{组合覆盖率}
\author{张磊}
\maketitle
\subsection*{组合覆盖率}

$i\_ways$覆盖率的定义:
\begin{equation}
C_i^{(n)}=\frac{m_i(n)}{M_i}
\end{equation}
\\其中$m_i(n)$代表第n个测试用例结束后,已经用到的总的$i\_ways$组合的个数,它是关于n的函数。$M_i$是总的$i\_ways$的组合个数。
\begin{equation}
Coverage_c(n)=C_1^{(n)} \cdot \frac{B_1(n)}{B(n)}+C_2^{(n)} \cdot \frac{B_2(n)}{B(n)}+C_3^{(n)} \cdot \frac{B_4(n)}{B(n)}+C_5^{(n)} \cdot \frac{B_i(n)}{B(n)}...+C_i^{(n)} \cdot \frac{B_i(n)}{B(n)}
\end{equation}
\\其中
\begin{equation}
$B(n)= \sum_{i=1}^N B_i(n) $
\end{equation}
\\$B_i(n)$为第n个测试用例结束后,对总的$i\_ways$的缺陷个数的估计值(因为开始时,不知道软件中的缺陷数,所以在测试的过程当中要对总的缺陷数进行估计),$B(n)$为系统当中总的缺陷数(这也是个估计值)。
\\其中$B_i(n)$的计算方式为:$\frac{B_i(n)}{b_i(n)}=\frac{M_i}{m_i(n)}$,由此可得
\begin{equation}
B_i(n)=\frac{M_i}{m_i(n)} \cdot b_i(n)
\end{equation}
\\其中$b_i(n)$是n个测试用例结束后前n个测试用例发现的$i\_ways$总的缺陷数。
\\将公式(1)(4)带入到公式(2)中可以化简得到
\begin{equation}
Coverage_c(n)=\frac{\sum_{i=1}^N b_i(n)}{B(n)}
\end{equation}
\\其中公式(5)的含义是,n个测试用例结束后,组合覆盖率=发现的缺陷的总的个数/估计的总的缺陷数
\\总结:通过以上文档可以看到,首先我们定义了$i\_ways$的覆盖率,然后我们定义了组合覆盖率,根据组合覆盖率的定义我们可以看出,组合覆盖率是通过$i\_ways$的覆盖率乘以其对应的权值,其中$i\_ways$的覆盖率的权值是$i\_ways$检测的缺陷数$B_i(n)$与我们估计的总的缺陷数$B(n)$的比值,这样以来我们就将组合覆盖率与缺陷的发现率联系到了一起,就像我们公式(5)定义的那样。
\end{CJK}
\end{document}