\documentclass[10pt,journal,letterpaper,compsoc]{IEEEtran}

%%%%%%%%%%%%%%%%%%%%%%%%%%
\usepackage[ruled]{algorithm2e}
\renewcommand{\algorithmcfname}{ALGORITHM}
\SetAlFnt{\small}
\SetAlCapFnt{\small}
\SetAlCapNameFnt{\small}
\SetAlCapHSkip{0pt}
\IncMargin{-\parindent}

%\usepackage{mathbbold}
\usepackage{amsfonts}

%---------------------------usepackages----------------------------
\usepackage[width=.45\textwidth]{caption}
\usepackage{cite}
\usepackage[dvips]{graphicx}
\usepackage{algpseudocode}
\usepackage{array}
\usepackage{float}
\usepackage[tight,footnotesize]{subfigure}
\usepackage{booktabs}
\usepackage{url}
\usepackage{multirow}
\newcommand\du{\mathrm{d}}
\usepackage[cmex10]{amsmath}
\usepackage{amssymb}
\usepackage{mathrsfs}
\usepackage{CJKutf8}
\usepackage[margin=2cm]{geometry}
\usepackage[utf8]{inputenc}

%---------------------------THEOREMs------------------------------
\newtheorem{thm}{Theorem}[section]

\newtheorem{definition}{Definition}
\newtheorem{lemma}[thm]{Lemma}
\newtheorem{claim}[thm]{Claim}
\newtheorem{corol}[thm]{Corollary}
\newtheorem{propos}[thm]{Proposition}
\newtheorem{rema}{Remark}[section]
\def\bde{\begin{definition}}
\def\ede{\end{definition}}
\def\bp{\begin{propos}}
\def\ep{\end{propos}}
\def\bt{\begin{thm}}
\def\et{\end{thm}}
\def\bco{\begin{corol}}
\def\eco{\end{corol}}
\def\bl{\begin{lemma}}
\def\el{\end{lemma}}
\def\br{\begin{rema}}
\def\er{\end{rema}}
\def\be{\begin{equation}}
\def\ee{\end{equation}}
\def\ba{\begin{array}}
\def\ea{\end{array}}
\def\bena{\begin{eqnarray}}
\def\eena{\end{eqnarray}}

%-------------------------------Letter ab.------------------------
\def\P{{\mathbb P}}
\def\E{{\mathbb E}}
\def\R{{\mathbb R}}
\def\Z{{\mathbb Z}}
\def\N{{\mathbb N}}
\def\Q{{\mathbb Q}}
%\def\1{\mathbbold{1}}
\def\1{I}
%--------------------
\def\fA{{\cal A}}
\def\fB{{\cal B}}
\def\fC{{\cal C}}
\def\fD{{\cal D}}
\def\fE{{\mathscr E}}
\def\fF{{\mathscr F}}
\def\fG{{\mathscr G}}
\def\fK{{\mathscr K}}
\def\fL{{\mathscr L}}
%--------------------
\def\ze{{\zeta}}
\def\nb{{\nonnumber}}
\def\lb{{\label}}
\def\la{{\lambda}}\def\La{{\Lambda}}
\def\ga{{\gamma}}\def\Ga{{\Gamma}}
\def\a{{\alpha}}
\def\d{{\delta}}\def\D{{\Delta}}
\def\o{{\omega}}\def\O{{\Omega}}
\def\var{\hb{Var}}

%------------------------------OTHERS------------------------------------
\def\QED{\hfill$\square$\vskip 3mm}
\def\qed{\vskip -22pt \QED}
\def\Dp{\displaystyle}
\def\Df{\Dp\frac}
\def\hb{\hbox}
\def\({\left(}
\def\){\right)}
\def\[{\left[}
\def\]{\right]}
\linespread{1}
 \date{}
\begin{document}
%\large
\begin{CJK}{UTF8}{gkai}

\title{数学公式推导}
\author{张磊}
\maketitle
\section{Algorithm 1}
 \begin{thm}
Let $B_i$ represent i\_ways faults, and $M_i$ represent the number of i\_ways combinations.
Let $b_i(n)$ denote the number of faults for i\_ways combination when the $n$ test cases finished testing, 
and  $m_i(n)$ denote the used number of i\_ways combinations when the $n$ test cases finished testing.
The counting process $b_i(n),m_i(n)$ satisfy the following :
$\forall\varepsilon>0$,  
\begin{equation}  
\P(|\frac{b_i(n)}{m_i(n)} - \frac{B_i}{M_i}| \le \varepsilon )
        \geq {1-\frac{1}{m_i(n)\varepsilon}}
\end{equation}
\end{thm}

\begin{proof}
When $n\rightarrow \infty$, it is clear that $ m_i(n)\rightarrow M_i $, and $ b_i(n)\sim \binom{m_i(n)}
{\frac{B_i}{M_i}} $, Chebyshev's inequality gives 

$ 1 \geq P \{\vert \frac{b_i(n)}{m_i(n)} - 
\frac{B_i}{M_i}\vert < \varepsilon \}
\geq 1-\frac{m_i(n)\cdot \frac{B_i}{M_i} \cdot  (1-\frac{B_i}{M_i})}{m_i(n)^2\cdot \varepsilon^2} $
\\Moreover,
$1-\frac{m_i(n)\cdot \frac{B_i}{M_i} \cdot  (1-\frac{B_i}{M_i})}{m_i(n)^2\cdot \varepsilon^2}
\geq 1-\frac{1}{m_i(n) \cdot \varepsilon} $, so that the proof holds. 
\end{proof}

\begin{thm}
Assume that $b_i^{m_1}(n)$  and $b_i^{m_2}(n)$ are the total number of found faults under the same optimized weight
when the number of $n$ test cases completed, and the parameters $m$ equal  
$m_1,m_2(m_1<m_2)$. Then, 
\begin{equation}  
E(b_i^{m_1}(n)) \leq E(b_i^{m_2}(n)) 
\end{equation}
\end{thm}

\begin{proof}
The Mathematical Induction (MI) is used here. 
\begin{enumerate}
\item when $n=1$, the best test case is selected from the number of $m_2$ test cases,
    and the number of detected bugs is $E(b_i^{m_2}(1))$. This process can viewed as
    the $m_1$ test cases is firstly selected, in which the best test case is selceted to  
    detect  $E(b_i^{m_1}(1))$  bugs. Furthermore, the  $m_2-m_1$ test cases are used, 
    in which the best test case is selected to detect $E(b_i^{(m_2-m_1)}(1))$ bugs. 
    Since $E(b_i^{m_2}(1))=E(max\{b_i^{m_1}(1),b_i^{(m_2-m_1)}(1)\})$, so it is clear that
    $E(b_i^{m_1}(1)) < E(b_i^{m_2}(1))$
\item when $n\leq k$, $E(b_i^{m_1}(k)) \leq E(b_i^{m_2}(k))$. If $n=k+1$,
    the negation approach is used to prove $E(b_i^{m_1}(k+1)) \leq E(b_i^{m_2}(k+1))$.
\item We first assume that $E(b_i^{m_1}(k+1)) > E(b_i^{m_2}(k+1))$, when $m=m_1$,
    the detected faults is $X_k$ and $Y_k$ using the $k^{th}$ test case. 
    Then, $b_i^{m_1}(k)=\sum_{n=1}^{k}X_n$, and $b_i^{m_2}(k)=\sum_{n=1}^{k}Y_n$.
    We assume that when $m=m^1$ and $m=m^2$,the $(k+1)^{th}$ selected best test case detect the number of expected bugs,
    which are $E(X(k+1)),E(Y(k+1))$. 
    \begin{enumerate}
    \item Since we assume $E(b_i^{m_1}(k+1)) > E(b_i^{m_2}(k+1))$,i.e., 
          $E(b_i^{m_1}(k))+E(X(k+1)) > E(b_i^{m_2}(k))+E(Y(k+1))$. 
          We can assume that two virtulized selection processes exist in the $(k+1)^{th}$ selected best test case detecting the number of expected bugs $E(X(k+1))$.
          First we select test cases from $m^0$ and then $ m^1-m^0 $. Accordingly, the number of detected expected bugs are $E(X^{m_0}(k+1))$ and $E(X^{(m_1-m^1_0)}(k+1))$, respectively.
          In which, $E(X(k+1))=E(max \{X^{m^1_0}(k+1)),X^{(m_1-m^1_0)}(k+1))\} )$. 
          When $m=m^2$, we use the same approach to select test cases and then detec bugs. So that, 
          the number of detected expected bugs are $E(Y^{m_0}(k+1))$ and $E(Y^{(m_2-m_0)}(k+1))$, respectively.
          In which, 
          $E(Y(k+1))=E(max \{Y^{m^2_0}(k+1),Y^{(m_2-m^2_0)}(k+1))\})$ 
    \item We know from the sampling theory there exist a $m_0$, which make   
          $E(b_i^{m_1}(k))+X(k+1) = E(b_i^{m_2}(k))+Y(k+1)$. Therefore, the second selection process during the $k_{th}$ are from $m^1-m^0$,$m^2-m^0$.
          Since $m^1-m^0<m^2-m^0$, 
          so that $X^{(m_2-m^2_0)}(k+1)) \leq Y^{(m_2-m^2_0)}(k+1))$. And furthermore, 
           $E(X(k+1)) =E(max \{X^{m^1_0}(k+1)),X^{(m_1-m^1_0)}(k+1))\} )$,
           $E(Y(k+1)) =E(max \{Y^{m^1_0}(k+1)),Y^{(m_1-m^1_0)}(k+1))\} )$. And Because, 
           $E(b_i^{m_1}(k))+X^{m_0}(k+1)) = E(b_i^{m_2}(k))+Y^{m_0}(k+1))$, i.e.,
           $E(b_i^{m_1}(k)) = E(b_i^{m_2}(k))+ (Y^{m_0}(k+1)) - X^{m_0}(k+1)))$.
           And also, 
           $(E(Y(k+1)) - E(X(k+1)))
           \geq (Y^{m_0}(k+1)) - X^{m_0}(k+1)))$
          Thus, $E(b_i^{m_1}(k)) \leq  E(b_i^{m_2}(k))+(Y(k+1) - X(k+1))$
          i.e., $E(b_i^{m_1}(k))+ X(k+1)) \leq E(b_i^{m_2}(k))+Y(k+1))$
          i.e., $ E(b_i^{m_1}(k+1)) \leq E(b_i^{m_2}(k+1))$
          This will be paradox with the assumption  $E(b_i^{m_1}(k+1)) > E(b_i^{m_2}(k+1))$. Therefore, the proof holds true. 
    \end{enumerate}
\end{enumerate}
\end{proof}
The meaning of theorem 2 :The algorithm 1 find the expected faults after execute the same test case by applying parameter $m^1$,$m^2$($m^1<m^2$).
The larger value of $m$, the more effective for the detection.

\section{Algorithm 2}
The theories of algorithm 1 apply the alogorithm 2 and 3. 
The updating weight 1:
\begin{thm}
\begin{equation} 
\rho _i(k+1)= \frac{\rho_i(k)+\frac {b_i(k+1)}{m_i(k+1)}}{2} \rightarrow  \frac{B_i}{M_i}
\end{equation}
\end{thm}

\begin{proof}
$\rho_i(1)= \frac{\rho_i(0)+\frac {b_i(1)}{m_i(1)}}{2}
=\frac{1}{2}\rho_i(0)+\frac{1}{2}\frac{b_i(1)}{m_i(1)}$
\\$\rho_i(2)= \frac{\rho_i(1)+\frac {b_i(2)}{m_i(2)}}{2}
=\frac{1}{2^2}\rho_i(0)+\frac{1}{2^2}\frac{b_i(1)}{m_i(1)}
+\frac{1}{2}\frac{b_i(2)}{m_i(2)}$.
And also, we obtain  $\rho_i(n)= \frac{\rho_i(n-1)+\frac {b_i(n)}{m_i(n)}}{2}
=\frac{1}{2^n}\rho_i(0)+\frac{1}{2^n}\frac{b_i(1)}{m_i(1)}
+\frac{1}{2^{n-1}}\frac{b_i(2)}{m_i(2)}+……+\frac{1}{2}\frac{b_i(n)}{m_i(n)}$.
From the following lemma we conclude $\rho_i(n)\rightarrow\frac{B_i}{M_i}$
\end{proof}

\begin{lemma}
For sequences $P_1,P_2.....P_n$
satisfying $ \lim \frac{P_n}{P_1+P_2+......+P_n} \rightarrow 0 $.
If $\lim a_n \rightarrow a$,
then $\lim \frac{a_n \cdot P_1+a_{n-1} \cdot P_2+...+a_1\cdot P_n}{P_1+P_2+......+P_n}=a$
\end{lemma}

\begin{proof}
Because $\lim a_n \rightarrow a$
so that $\forall \varepsilon,\exists N_1$,
for all $n>N_1$, there exsits an equation 
$|a_n-a|<\frac{\varepsilon}{3}$,
and also $\{a_n\}$ is bounded, there $\exists M>0$, which makes $|a_n|<M$.
For a sequence $\{P_1,P_2.....P_n\}$,
satisfying $\lim \frac{P_n}{P_1+P_2+......+P_n} \rightarrow 0$.
For all the above $\varepsilon$,there exists $N_2$,which make when $n>N_2$,
$|\frac{P_n}{P_1+P_2+......+P_n}-0| \leq \frac{\varepsilon}{3MN_1}$,
pick up $N_3>max\{N_1,N_2\}$,
and make three segments for $n$, so that we have 
\begin{eqnarray*}
| \frac{a_n P_1+a_{n-1} P_2+...+a_1 P_n}{P_1+P_2+......+P_n}-a| \\
=( \frac{|a_1-a| P_n+|a_2-a| P_{n-2}+...+|a_n-a| P_1}{P_1+P_2+......+P_n})\\
\leq ( \frac{|a_1-a| P_n+|a_2-a| P_{n-2}+...+|a_{N_1}-a| P_{n-N_1+1}}{P_1+P_2+......+P_n})\\
+ ( \frac{|a_{N_1+1}-a| P_{n-N_1}+...+|a_{n-N_3}-a|  P_{N_3+1}}{P_1+P_2+......+P_n})\\
+ ( \frac{|a_{n-N_3+1}-a|  P_{N_3}+...+|a_{n}-a| P_1}{P_1+P_2+......+P_n})       
\end{eqnarray*}
\\Since ${a_n}$ is bounded,$\forall n,\exists M>0$
makes $|a_n-a|<M$
\begin{eqnarray*}
( \frac{|a_1-a| P_n+|a_2-a| P_{n-2}+...+|a_{N_1}-a| P_{n-N_1+1}}{P_1+P_2+......+P_n})\\
\leq M.\frac{P_n+...+P_{n-N_1+1}}{P_1+P_2+...+P_n} \leq N_1 \cdot M \cdot \frac{\varepsilon}{3MN_1}
=\frac{\varepsilon}{3}
\end{eqnarray*}
\begin{eqnarray*}
( \frac{|a_{N_1+1}-a| P_{n-N_1}+...+|a_{n-N_3}-a|  P_{N_3+1}}{P_1+P_2+......+P_n})\\
    \leq(\frac{P_{n-N_1}+...+P_{N_3+1}}{P_1+P_2+...+P_n}
\cdot \frac{\varepsilon}{3})
    \leq \frac{\varepsilon}{3}
\end{eqnarray*}
\begin{eqnarray*}
( \frac{|a_{n-N_3+1}-a|  P_{N_3}+...+|a_{n}-a| P_1}{P_1+P_2+......+P_n}) \leq \\
(\frac{P_{N_3}+...+P_1}{P_1+P_2+...+P_n} \cdot \frac{\varepsilon}{3}) \leq \frac{\varepsilon}{3}
\end{eqnarray*}
\begin{equation}  
 \forall n>N_3, | \frac{a_n P_1+a_{n-1} P_2+...+a_1 P_n}{P_1+P_2+......+P_n}-a|\leq \varepsilon 
\end{equation}
\\We conclude this lemma. 
\end{proof}

\subsection{定理6:}
权值调整方法2:
\begin{equation}  
 \begin{aligned}
 & \rho_i(k+1)=\frac{1\cdot\rho_i(0)+...+k\cdot\rho_i(k-1)}
   {\frac{(k+2)(k+3)}{2}}+\\
 & \frac{(k+1)\cdot\rho_i(k)+ (k+2)\cdot\frac{b_i(k+1)}{m_i(k+1)}}
   {\frac{(k+2)(k+3)}{2}}\\
&\rho_i(k+1)\rightarrow \frac{B_i}{M_i}
 \end{aligned}
\end{equation}
\\证明过程:
\\$\rho_i(1)=\frac{\rho_i(0)+2\frac{b_i(1)}{m_i(1)}}{\frac{2.3}{2}}=\frac{1}{3}\rho_i(0)+\frac{2}{3}\frac{b_i(1)}{m_i(1)}=a_i^1(0)\rho_i(0)+a_i^1\frac{b_i(1)}{m_i(1)}$
\\$\rho_i(2)=\frac{\rho_i(0)+2\rho_i(1)+3\frac{b_i(1)}{m_i(1)}}{\frac{3.4}{2}}=a_i^2(0)\rho_i(0)+a_i^2\frac{b_i(1)}{m_i(1)}+a_i^2(2)\frac{b_i(2)}{m_i(2)}$
\\设$\rho_i(n)=a_i^n(0)\rho_i(0)+a_i^n(1)\frac{b_i(1)}{m_i(1)}+a_i^n(2)\frac{b_i(2)}{m_i(2)}+...+a_i^n(n)\frac{b_i(n)}{m_i(n)}$
\\其中参数经过数学计算处理得到:
$ \begin{displaymath}
   a_i^n(k) = \left\{
     \begin{array}{lr}
       \frac{2}{k+2}  & n = k\\
       \frac{4(k+1)}{(k+2)^2(k+3)} & n=k+1 \\
       \frac{n+3}{n+1}\frac{4(k+1)}{(k+4)(k+3)(k+2)} & n>k+1
     \end{array}
   \right.
\end{displaymath}$
\\令$P_k=a_i^n(n-k)$
\\则$\rho_i(n)=P_n\rho_i(0)+P_{n-1}\frac{b_i(1)}{m_i(1)}+P_{n-2}\frac{b_i(2)}{m_i(2)}+...+P_0\frac{b_i(n)}{m_i(n)}$
\\其中系数$P_1,P_2...P_n$满足$\lim \frac{P_n}{P_1+P_2+......+P_n} \rightarrow 0$,即$\rho_i(n)$满足引理四,所以有$\rho_i(n)\rightarrow \frac{B_i}{M_i}$

\subsection*{定理六具体证明过程:}
\begin{equation}  
\rho_i(k+1)=\frac{1\cdot\rho_i(0)+...+k\cdot\rho_i(k-1)+(k+1)\cdot\rho_i(k)+(k+2)\cdot\frac{b_i(k+1)}{m_i(k+1)}}{\frac{(k+2)(k+3)}{2}}
\end{equation}
\\证明过程:
\\$\rho_i(1)=\frac{\rho_i(0)+2\frac{b_i(1)}{m_i(1)}}{\frac{2.3}{2}}=\frac{1}{3}\rho_i(0)+\frac{2}{3}\frac{b_i(1)}{m_i(1)}=a_i^1(0)\rho_i(0)+a_i^1\frac{b_i(1)}{m_i(1)}$
\\$\rho_i(2)=\frac{\rho_i(0)+2\rho_i(1)+3\frac{b_i(1)}{m_i(1)}}{\frac{3.4}{2}}=a_i^2(0)\rho_i(0)+a_i^2\frac{b_i(1)}{m_i(1)}+a_i^2(2)\frac{b_i(2)}{m_i(2)}$
\begin{equation}
\rho_i(n)=a_i^n(0)\rho_i(0)+a_i^n(1)\frac{b_i(1)}{m_i(1)}+a_i^n(2)\frac{b_i(2)}{m_i(2)}+...+a_i^n(n)\frac{b_i(n)}{m_i(n)}
\end{equation}
\\将(7)公式中的n分别取值0,1,2,3.......k,分别得到
$\rho_i(0),\rho_i(1),\rho_i(2),\rho_i(3).....\rho_i(k+1)$,
然后带入到公式(6)中,并得到等号两边对应系数相等,
$\rho_i(0)$的系数对应相等将会得到
$a_i^n(0)=\frac{1+2a_i^1(0)+3a_i^2(0)+...+na_i^{n-1}(0)}{\frac{(n+1)(n+2)}{2}}$
\\${\frac{(n+1)(n+2)}{2}}a_i^n(0)={1+2a_i^1(0)+3a_i^2(0)+...+na_i^{n-1}(0)}$
\\将上式的n取n+1将会得到
\\${\frac{(n+2)(n+3)}{2}}a_i^{n+1}(0)={1+2a_i^1(0)+3a_i^2(0)+...+na_i^{n-1}(0)+(n+1)a_i^n(0)}$ 
\\将上边两式做减法将会得到
\\${\frac{(n+2)(n+3)}{2}}a_i^{n+1}(0)-{\frac{(n+1)(n+2)}{2}}a_i^n(0)=(n+1)a_i^n(0)$
\\整理得到$a_i^{n+1}(0)=\frac{(n+1)(n+4)}{(n+2)(n+3)}a_i^n(0)$
\\$a_i^1(0)=\frac{1}{3},a_i^2(0)=\frac{2 \cdot 5}{3 \cdot 4} \cdot \frac{1}{3}$
\\$a_i^3(0)=\frac{3 \cdot 6}{4 \cdot 5 } \cdot \frac{2 \cdot 5}{3 \cdot 4} \cdot \frac{1}{3}
=\frac{6}{4} \cdot \frac{2}{4} \cdot \frac{1}{3}$
\\$a_i^4(0)=\frac{4 \cdot 7}{5 \cdot 6 } \cdot \frac{6 \cdot 2}{4 \cdot 4} \cdot \frac{1}{3}
=\frac{7}{5} \cdot \frac{2}{4} \cdot \frac{1}{3}$
\\同理得
\\$a_i^n(0)=\frac{(n+3)}{(n+1)} \cdot \frac{2}{4} \cdot a_i^1(0)$
\\同上,再处理$a_i^n(1)$

$\frac{b_i(1)}{m_i(1)}$的系数对应相等将会得到
\\$a_i^n(1)=\frac{2a_i^1(1)+3a_i^2(1)+...+na_i^{n-1}(1)}{\frac{(n+1)(n+2)}{2}}$
\\${\frac{(n+1)(n+2)}{2}}a_i^n(1)={2a_i^1(1)+3a_i^2(1)+...+na_i^{n-1}(1)}$
\\同样将上式n取n+1得到
\\$\frac{(n+2)(n+3)}{2}a_i^{n+1}(1)=2a_i^1(1)+3a_i^2(1)+...+na_i^{n-1}(1)+(n+1)a_i^{n}(1)$
\\两式相减得到$a_i^{n+1}(1)=\frac{(n+1)(n+4)}{(n+2)(n+3)}a_i^n(1)$
\\$a_i^1(1)=\frac{2}{3},a_i^2(1)=\frac{2a_i^1(1)}{\frac{3 \cdot 4}{2}}=\frac{2}{9}$
\\$a_i^3(1)=\frac{3 \cdot 6}{4 \cdot 5 }  \cdot \frac{2}{9}
=\frac{1}{5}$
\\$a_i^4(1)=\frac{4 \cdot 7}{5 \cdot 6 } \cdot \frac{1}{5}$
\\同理得到:
\\$a_i^n(1)=\frac{3}{5} \cdot \frac{(n+3)}{(n+1)} \cdot \frac{2}{9} \cdot a_i^2(1)
=\frac{2}{15} \cdot \frac{(n+3)}{(n+1)} \cdot a_i^2(1)$
\\再用同样的方式处理$a_i^n(2)$,最后得到
\\$a_i^n(1)=\frac{(n+3)}{(n+1)} \cdot \frac{4}{6} \cdot a_i^3(2)$
\\递归得出
\\$a_i^n(k)=\frac{(n+3)}{(n+1)} \cdot \frac{(k+2)}{k+4} \cdot a_i^{k+1}(k)$

即
$ \begin{displaymath}
   a_i^n(k) = \left\{
     \begin{array}{lr}
       \frac{2}{k+2}  & n = k\\
       \frac{4(k+1)}{(k+2)^2(k+3)} & n=k+1 \\
       \frac{n+3}{n+1}\frac{4(k+1)}{(k+4)(k+3)(k+2)} & n>k+1
     \end{array}
   \right.
\end{displaymath}$

\subsection*{错误权值更新方式:}
\begin{equation}
\rho_i(k+1)=\frac{\rho_i(0)+\rho_i(1)+\rho_i(2)+\rho_i(3)+....\rho_i(k)+\frac{b_i(k+1)}{m_i(k+1)}}{(k+2)}
\end{equation}
$\\ \rho_i(1)=\frac{\rho_i(0)+\frac{b_i(1)}{m_i(1)}}{2}
=\frac{1}{2} \cdot \rho_i(0)+\frac{1}{2} \cdot \frac{b_i(1)}{m_i(1)}
=a_i^1(0)\rho_i(0)+a_i^1(1) \frac {b_i(1)}{m_i(1)}
\\ \rho_i(2)=\frac{\rho_i(0)+\rho_i(1)+\frac{b_i(2)}{m_i(2)}}{3}
=a_i^2(0)\rho_i(0)+a_i^2(1) \frac {b_i(1)}{m_i(1)}+a_i^2(2)\frac{b_i(2)}{m_i(2)}
\\ \rho_i(3)=\frac{\rho_i(0)+\rho_i(1)+\rho_i(2)\frac{b_i(3)}{m_i(3)}}{4}
=a_i^3(0)\rho_i(0)+a_i^3(1) \frac {b_i(1)}{m_i(1)}+a_i^3(2)\frac{b_i(2)}{m_i(2)}+a_i^3(3)\frac{b_i(3)}{m_i(3)}$
\\由上可得:
\begin{equation}
\rho_i(n)=\frac{\rho_i(0)+\rho_i(1)+...+\rho_i(2)\frac{b_i(n)}{m_i(n)}}{(n+1)}
=a_i^n(0)\rho_i(0)+a_i^n(1) \frac {b_i(1)}{m_i(1)}+a_i^n(2)\frac{b_i(2)}{m_i(2)}+a_i^n(3)\frac{b_i(3)}{m_i(3)}+....+a_i^n(n)\frac{b_i(n)}{m_i(n)}
\end{equation}
\\将上述公式中n分别取0,1,2....k+1,带入到公式(8)中,将会得到等式两边对应的
$\\ \frac{b_i(1)}{m_i(1)},\frac{b_i(2)}{m_i(2)},\frac{b_i(3)}{m_i(3)}....\frac{b_i(1)}{m_i(k+1)}$
系数分别相等\\首先处理
$a_i^n(0),a_i^0(0)=\frac{1}{2},a_i^1(0)=\frac{1}{2}
\\a_i^n(0)=\frac{a_i^0(0)+a_i^1(0)+a_i^2(0)+...+a_i^{n-1}(0)}{(n+1)}
\\{(n+1)}a_i^n(0)={a_i^0(0)+a_i^1(0)+a_i^2(0)+...+a_i^{n-1}(0)}
\\{(n+2)}a_i^{n+1}(0)={a_i^0(0)+a_i^1(0)+a_i^2(0)+...+a_i^n(0)}$
\\上边两式相减得到
$a_i^{n+1}(0)=a_i^n(0)=\frac{1}{2}$
\\再处理
$a_i^n(1)$
$a_i^n(1)=\frac{a_i^1(1)+a_i^2(1)+...+a_i^{n-1}(1)}{(n+1)}$
\\${(n+1)}a_i^n(1)=a_i^1(1)+a_i^2(1)+...+a_i^{n-1}(1)$
\\${(n+2)}a_i^{n+1}(1)=a_i^1(1)+a_i^2(1)+...+a_i^{n}(1)$
\\上边两式相减,得到
$a_i^{n+1}(1)=a_i^n(1)=\frac{1}{6}$
\\用同样的方法我们可以求得
$a_i^{n+1}(n-1)=\frac{a_i^n(n-1)}{n+1}=\frac{1}{(n+1) \cdot n}
\\a_i^n(n)=\frac{1}{n+1}$
\\所以有公式(9)其中
$\\a_i^n(0)=\frac{1}{1 \cdot 2},a_i^n(1)=
\frac{1}{2 \cdot 3},a_i^n(2)=\frac{1}{3 \cdot 4}....a_i^n(n-1)
=\frac{1}{n \cdot (n+1)},a_i^n(n)=\frac{1}{n+1}
\\p(k)=a_i^n(n-k)
\frac{b_i(n)}{m_i(n)} \rightarrow \frac{B_i}{M_i}
\frac{P(n)}{P(1)+P(2)+...+P(n)}=
\frac{\frac{1}{2}}{\frac{1}{1 \cdot 2}+\frac{1}{2 \cdot 3}+\frac{1}{3 \cdot 4}+...
+\frac{1}{n \cdot n+1}+\frac{1}{1 \cdot n}} \rightarrow \frac{1}{2}$
\\所以这种权值更新的方法并不能符合引理四的初始条件,所以这种更新权值的方式是错误的
\end{CJK}
\end{document}
